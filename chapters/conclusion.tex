%[This is the conclusion.Here I sum up all the crucial points of my work, from the vision, to my results, to some of the major points in the discussion.]
Visible light communication is a good candidate to achieve local, situated communication.
These two properties combined make it a suitable form of communication to be used in the context of mobile robotics, to achieve robot-to-robot and environment-to-robot communication.
The development and evaluation of a prototype system using general purpose hardware and platforms demonstrate how easily communication can be achieved within certain boundaries.
The prototype can achieve local communication within 30 cm of range in direct line of sight source-to-destination.
If mounted on a robot, the system is able to identify the light source from farther distances and with different angles, although reception of messages has not been established from afar.
The mobile robot however can be programmed to approach the light source, so that when it comes in range, messages are correctly received.
The minimum rate for the communication to be successful is of about 500 Hz with the given hardware setup, but lower rates achieve higher accuracy.
Even at this low rates, errors in reception are still present, but a simple protocol allows the receiver to acknowledge when they occur.
Since only mono-directional communication has been established, the transmitter sends the same message repeatedly.

Hardware characteristics of the system can potentially help increase the performance of the communication in most measurable ways.
For this reason, VLC systems can be extremely flexible and adaptive, a choice on hardware components can influence many aspects of the communication as the application requires.
Some of the strong points of visible light communication are that light emitting devices like LEDs are generally cheap, available, human friendly and with low power consumption.
As a plus, light is not subject to electromagnetic interference, exploiting an entirely different band in the spectrum.
In the context of mobile robotics, a VLC system can achieve local communication since light is easily restrained by hardware and physical barriers, and most importantly it allows localisation of the transmission source.

In this like in other contexts, this medium of transmission can exploit a large amount of light emitting devices already present everywhere to achieve environment-to-device communication, providing additional functionalities other than illumination first among which indoor positioning.

Visible light is not the only form of communication that can achieve this, also signals in the non-visible spectrum have similar properties, like infra red and ultra violet.
Additionally, non situated forms of communication can be extended with a situated component to form a hybrid system, or adjusted with techniques that can allow the source of the communication to be identified and localised while exchanging generic information.
Visible light however can be recognised by the human eye, allowing an immediate way for humans to determine if the system is running in a correct way and to interact with it.
In some applications, a communication system using visible light could also provide illumination as an additional feature, other than just communication.

