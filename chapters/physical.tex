\section{Physical layer}
\label{physical}
This layer is about physical properties of the hardware and the signal itself.\\
In this section these properties will be analysed through the aid of various experiments performed on the prototype system.
%motivation
This is a first step in understanding the potential and the limitations of a system, because whatever limitation derives from the physical layer, it will be a hard limitation that will affect the entire system.
Therefore, starting to look at physical constraints of the system will give a good overview of what can be achieved and what cannot early on, before the rest of the designing process.
% experimental setup
All the following experiments will be performed on the prototype system as described in section \ref{expsetup}, with the use of the low power 10mm LED unless otherwise stated.
%why brigthness in %
A remark on the experiments that will follow is that brightness will be always reported in percentage.
This serves the purpose of making the data applicable in general cases, even with different light sources.
The LED used for testing produces a luminous intensity of about 8 000 to 10 000 MCD at 25$^{\circ}$ \todo{reference datasheet for led}.

%warmup
\subsection{Warmup time}

%These are stats for warmup only at dark
\begin{figure}[h]
\centering
\includegraphics[height=140px]{img/warmup1}
\includegraphics[height=140px]{img/warmdown1}
\caption{LED warmup and cooldown times.}
\label{fig:warmup}
\end{figure}

The speed of light transmission depends in primis on the speed at which the light itself can be turned on and off.
Figure \ref{fig:warmup} shows the warmup times for the low power LED over multiple instances, meaning the time that it takes for turning the light completely on from completely off, and vice versa.
These measurements depend on the reception rate of the system, which will be discussed in section \ref{recrates}.
Table \ref{tab:warmup} shows the times for the LED to switch between specific brightness levels.
Each row represents the time to reach the level on each column, for example the first row represents the time to reach any brightness level starting from 0\% brightness.
The table works both ways, meaning it shows the time for the warmup as well as the time for the cool down of the LED.
As another example, the last row shows how long it takes to reach any level from a completely ON state, meaning starting from 100\% of brightness.

\begin{table}[hbt]
\centering
  \begin{tabular}{ l | c c c c c c c c c c c}
    & 0\% & 10\% & 20\% & 30\% & 40\% & 50\% & 60\% & 70\% & 80\% & 90\% & 100\% \\
    \hline
0\% & - & 0.3 & 0.66 & 1.08 & 1.52 & 2.12 & 2.82 & 3.78 & 5.38 & 10.4 & 50.66 \\
10\% & 55.88 & - & 0.36 & 0.78 & 1.22 & 1.82 & 2.52 & 3.48 & 5.08 & 10.1 & 50.36 \\
20\% & 61.86 & 5.98 & - & 0.42 & 0.86 & 1.46 & 2.16 & 3.12 & 4.72 & 9.74 & 50.0 \\
30\% & 63.76 & 7.88 & 1.9 & - & 0.44 & 1.04 & 1.74 & 2.7 & 4.3 & 9.32 & 49.58 \\
40\% & 64.9 & 9.02 & 3.04 & 1.14 & - & 0.6 & 1.3 & 2.26 & 3.86 & 8.88 & 49.14 \\
50\% & 65.76 & 9.88 & 3.9 & 2.0 & 0.86 & - & 0.7 & 1.66 & 3.26 & 8.28 & 48.54 \\
60\% & 66.4 & 10.52 & 4.54 & 2.64 & 1.5 & 0.64 & - & 0.96 & 2.56 & 7.58 & 47.84 \\
70\% & 66.98 & 11.1 & 5.12 & 3.22 & 2.08 & 1.22 & 0.58 & - & 1.6 & 6.62 & 46.88 \\
80\% & 67.48 & 11.6 & 5.62 & 3.72 & 2.58 & 1.72 & 1.08 & 0.5 & - & 5.02 & 45.28 \\
90\% & 67.92 & 12.04 & 6.06 & 4.16 & 3.02 & 2.16 & 1.52 & 0.94 & 0.44 & - & 40.26 \\
100\% & 68.28 & 12.4 & 6.42 & 4.52 & 3.38 & 2.52 & 1.88 & 1.3 & 0.8 & 0.36 & - \\
  \end{tabular}
  \centering
  \caption{Warmup times in [ms] of the LED, for specific levels of brightness. Row: from brightness, Column: to brightness.}
  \label{tab:warmup}
\end{table}

From the table as well as from fig. \ref{fig:warmup}, it can be seen that the LED is slightly faster at being turned on rather than off.
Also pretty indicative is the fact that about 90\% of the time used for turning a LED completely on or completely off, is spent to make a variation in the last 20\% of brightness levels, which means that a LED can reach a reasonably high brightness in a time relatively small compared to its full potential.
Interesting for later sections is that 50\% of the brightness can be reached in about 2 ms.
\todo{stats for big bulb?}

%ambient
\subsection{Interference from ambient light}
Previous data about times to reach maximum brightness were measured in a condition of full darkness of the environment surrounding the light source, at minimal distance between transmitter and receiver.
But visible light communication is not necessarily used with this restriction, it is in fact meant to be used over an open space, wirelessly,  meaning that interference from natural light is very likely.
 Experiments have been performed to quantify the influence of ambient light.
 This is done by comparing the difference between the brightness measured by the sensor when the transmitter is turned ON and the one when the transmitter is turned OFF.
 This difference will be measured without ambient light interference and used as a reference for the measurements that occur with ambient light interference.
 Since communication  will later rely on the quantification of light variations, it's important to establish if this variations will be substantially different with or without interference present.
 For these experiments, for ambient light it is meant light from any source that is not directed to the receiver but naturally permeates the environment surrounding the system.
 If the variation ON/OFF in a fully dark environment is represented as 100\%, the experiment shows that this difference is between 100\% and 120\% when ambient light is present, averaging at around 110\%. 
Measurements were taken at the same conditions of distance and angle.
This increase could be explained with a higher sensibility of the sensor when exposed to higher levels of base brightness.
Overall it is fair to deduce the ambient light interference doesn't pay a big role when in reduced amount and without distance limitations.


%distance
\subsection{Distance}
\label{distancephy}
Another factor that might affect the communication over light is the distance between emitter and sensor, closely bound with the brightness reachable from the light emitter and the presence and weight of light interference.
Intuitively, brighter lights will be visible from farther.
A low power LED cannot reach high levels of brightness, therefore it won't be visible at long distances.
Table \ref{tab:distancesphy} shows the results of different measurements taken at increasing distances sensor to light source.
The reference for maximum brightness is achieved with a distance of 0 cm between receiver and light source, with the two nearly touching.
Figure \ref{fig:distancephy} shows the measurements of the same experiments in a graphical manner.
In the figure, each colour represents a different distance, and each peak a different experiment.
The experiment was performed in a dark environment, minimising the risk of interference from other light sources.

\begin{table}[hbt]
\centering
  \begin{tabular}{c c}
    distance & max. brightness \\
    \hline
    0 cm & 100\% \\
    10 cm & 42.81\% \\
    20 cm & 14.28\% \\
    30 cm & 8.57\% \\
    40 cm & 5.00\% \\
    50 cm & 3.57\% \\
    100 cm & 2.14\% \\
  \end{tabular}
 \caption{Maximum brightness over different distances.}
  \label{tab:distancesphy}
\end{table}

\begin{figure}[hbt]
	\centering
      \includegraphics[height=180px]{img/distancephy}
  \caption{Brightness at different distances.}
  \label{fig:distancephy}
\end{figure}

From the table and figure it can be seen that the maximum brightness registered from the sensor drops clearly with distance.
This obviously makes reliable communication over long distances harder.
An important implication with this results is the role of noise in the reception.
Noise doesn't scale with the communication, meaning that if the maximum brightness perceived at a certain distance is 50\% of the brightness at 0 distance, the noise in this first case is not 50\% of the one used for reference, but stays constant.
This means that the lower the variation perceived, the more noise will play a role in the communication.
A remark on the experiments performed is that the LED used for testing is directional, with an optimal viewing angle of about 30$^{\circ}$ from the centre of the emitter. 
The receiver, at the different distances, was always placed in the optimal viewing point in relation to the light emitter direction.

%angle
\subsection{Angle}
In addition to the previous factors, also the angle of incidence between the light emitter and the light sensor might be of relevance when establishing communication between the two ends, under certain circumstances.
So far, in all of the communication tests performed on the system, the receiver was always in direct line of sight with the light emitter and centred to its focus.
However,  if the light source is directional or semi-directional, the angle of incidence would affect the ability of the sensor to register light variations.
This could also be the case for omni-directional light sources, when reflection of the light would be critical for it to reach the sensor.
For example, the light source could not be bright enough to reach reflecting surfaces or these could be absent.
In the case of a regular room as the location of the communication system,  the light source must be powerful enough to either reach the sensor directly with sufficient intensity, or to reach it indirectly by reflecting on the walls, ceilings or other surfaces.
Experiments have been performed with different angles to show how this factor affects overall reception with a directional light source.
The measurements have been taken at a distance of 10 cm between light source and sensor, in a otherwise dark environment.
\begin{table}[hbt]
\centering
  \begin{tabular}{c c}
    Angle & max.brightness \\
    \hline
    0$^{\circ}$ & 100\% \\
    40$^{\circ}$ & 60\% \\
    60$^{\circ}$ & 50\% 
  \end{tabular}
  \caption{Maximum brightness measured at different angles.}
  \label{tab:anglesphy}
\end{table}
These results suggest that the larger the angle of exposure, the less powerful the reception. 

%power ac vs dc
\subsection{Powerful lights and power source}
\label{acphy}
\todo{talk about AC vs DC, discussion about power vs brightness vs distance vs speed}
As can be implied from the previous results, brighter and faster light sources are preferred to achieve better communication, at preferably low angle. The brighter a light, the farther it can be perceived, and the faster it is the more it allows fast switching, which in turn allows faster communication.
Such lights have a higher power consumption than less performant ones.
If the vision is to have a smart environment with lights that serve the double purpose of lighting the environment for humans and transmit information to devices, the most natural thing would be to connect these lights to mains electricity like any other lamp.
This power source potentially allows the usage of much more powerful light bulbs then the low power LED used in the prototype.
In Europe, electric power supply for mains is of 230 V at a frequency of 50 Hz \todo{reference mains information.}.
A distinctive trait of this power source is that it uses Alternating Current (AC).
Alternating current periodically reverses direction whereas direct current (DC) always flows in the same direction.
AC voltage can be expressed by a sinusoidal function, which means the current will vary its intensity in time.
Current will be higher near the peak, and lower near the time of the periodic switch in direction. 
This behaviour is clearly represented in figure \ref{fig:warmupAC}, when the light is ON after the warmup in the figure on the left, or before the cool down in the figure on the right.
The ON/OFF difference when measuring the light intensity with the receiver is over 500\% bigger than the LED of the prototype system, with similar rise speeds.
However, the fluctuations of AC during the ON state make it harder to use it for communication using On Off Keying, since that modulation scheme relies on variations in intensity.
A way around this would be to convert AC to DC with a process called rectification \todo{reference recification}.
This involves the use of a capacitor connected between AC V$^{+}$ and Ground,  which lowers the amplitude of the AC significantly thus producing something very similar to DC.
Another problem to face however, as shown in the picture, is that these LED bulbs have a much slower fall time when switched off, which also would affect the switching frequency during transmission.

\begin{figure}[h]
\centering
\includegraphics[height=140px]{img/warmup2}
\includegraphics[height=140px]{img/warmdown2}
\caption{AC bulb warmup time, and phases of alternating current.}
\label{fig:warmupAC}
\end{figure}
\todo{remove outlier in the picture of ac?}
