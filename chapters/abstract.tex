Visible light communication is a communication mode established with variations of visible light properties like intensity or colour. 
Quick variations in light intensity can be ignored by the human eye, while still being perceived by machines equipped with specific sensors to achieve communication.
Choosing light as a means of communication also provides characteristics that are particularly useful for mobile robotic agents, such as the ability to restrict the communication locally and most importantly to localise the source of transmission in space.
A prototype system has been realised using general purpose hardware platforms, to evaluate the characteristics of communication with light and the application of such a system in the context of mobile robotics.
With the restrictions imposed by the hardware, light communication using the system can be established within 30 cm to the source, at a rate of about 500 bits/s.
A receiver module has been later placed on a mobile robot, allowing the robot to follow the light source and receive the messages once in range.
Although communication is not established successfully farther than a few centimetres, the light source can be localised from longer distances, depending on environmental conditions such as reflective surfaces and presence of other light sources.
This characteristic allows the robot to navigate the environment and approach the source of communication in order to receive 