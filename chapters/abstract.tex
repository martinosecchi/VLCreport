%Visible light communication (VLC) is a communication mode established with variations of visible light properties like intensity or colour. 
%Quick variations in light intensity can be ignored by the human eye, while still being perceived by machines equipped with specific sensors, and can be used to exchange information between two parts.
Light as a means of communication provides characteristics that are particularly useful for mobile robotic agents.
In addition to normal exchange of messages, the source of transmission can be localised in space by the receiver.
A communication system including this feature facilitates processes such as distributed coordination, exploration and navigation, typical of swarm robotics.
A prototype system has been realised using general purpose hardware platforms, to evaluate the characteristics of communication with light and the application of such a system in the context of mobile robotics.
Restrictions and characteristics of the system have been evaluated at different levels of abstraction, from physical properties of the signal to the software techniques employed.
With the limitations affecting the system, communication using light can be established within 30 cm to the source, at a rate of about 500 bits/s.
To test the performance of the system in a potential robotic application, a receiver module has been later placed on a mobile robot, allowing the robot to follow the light source and receive the messages once in range.
Although reliable communication becomes harder with distance, the light source can be localised from farther away, depending on environmental conditions such as reflective surfaces and presence of other light sources.
Systems implementing visible light communication are very flexible in lack of strict standards and provided with the large variety of light emitting devices available.
A choice on the hardware components can make a system fit the specific requirements of different applications, directly affecting characteristics of transmission and reception such as rate, distance and viewing angle.
Conclusively, VLC appears to be a good candidate for achieving simultaneously local generic communication and mutual localisation between the parts involved, allowing complex messages to be exchanged between the parts.